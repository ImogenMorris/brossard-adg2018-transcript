\documentclass{beamer}

\usepackage{url} % for url
\usepackage[UKenglish]{babel} % for correct hyphenation patterns
\usepackage{amsfonts}
\usepackage{amsthm}
\usepackage{amsmath} % for equation* and bmatrix environment
\usepackage{IEEEtrantools} % for typesetting equations
\usepackage{xcolor} % for diagram colour
\usepackage[all,color]{xy} % for diagram
\usepackage{graphicx} % for images
\usepackage{tikz}
\usepackage{animate}
\usepackage{changepage} %for adjustable margins


% needed packages for Isabelle markup
\usepackage{verbatim}
\usepackage{pstricks,pst-node}
\usepackage{fancyvrb}
\usepackage[fancyvrb]{listings}
\usepackage{textcomp} %for ' quote
\usepackage[T1]{fontenc} %for " quote
\usepackage{stmaryrd} % for rbrakk
\usepackage{upquote} % for ` backtick
\usepackage{color}
\usepackage{isabella}

\usefonttheme{serif}

\setbeamertemplate{itemize items}[circle]
\setbeamertemplate{itemize subitem}{--}
\setbeamertemplate{bibliography item}[text]

\setbeamercolor*{structure}{bg=white,fg=black}

\setbeamertemplate{frametitle continuation}[from second]
%for italicised quote
\newenvironment{itquote}
  {\begin{quote}\itshape}
  {\end{quote}\ignorespacesafterend}
\newenvironment{itpars}
  {\par\itshape}
  {\par}

 \newcommand{\btVFill}{\vskip0pt plus 1filll}
 
%Information to be included in the title page:
\title{Towards a Mechanisation in Isabelle of Birkhoff's Ruler and Protractor Geometry \\ \textit{Transcript slides}}
\author{Imogen I. Morris and Jacques D. Fleuriot \\
s1402592@ed.ac.uk and jdf@inf.ed.ac.uk}

\institute{The University of Edinburgh}
\date{ADG 2018, Nanning}
 
 
 \usenavigationsymbolstemplate{}
 
\begin{document}

\begin{frame}
These transcript slides are available at this link:

\url{https://github.com/ImogenMorris/brossard-adg2018}
\url{-transcript}
\end{frame}

\begin{frame}[plain]
\graphicspath{{/Users/Imogen/Desktop/Birkhoff_Presentation/}}
    \makebox[\linewidth]{\includegraphics[width=\paperwidth]{ruler_prot.pdf}}
\end{frame}

\begin{frame}[plain]
 The first and most famous axiomatic theory of geometry is Euclid's `Elements'. Much of the geometry in the `Elements' can be described by two tools: the straight-edge and the compass. Proofs correspond to constructions with these tools, making them intuitive and visual. 
 
 However, some simple problems in Euclidean geometry cannot be solved by using only the straight-edge and compass. If we want to divide an angle into three equal angles, this is not possible using just the straight-edge and compass. It is also impossible to construct a square with the same area as a circle. This is equivalent to constructing a line segment with length $\pi$.
 
 If we have a measure or metric for lengths and angles, we can solve these problems. We can automatically obtain a line segment with length $\pi$ or length $e$. We can construct an angle of size $1$ or $\frac{1}{3}$. Reasoning also becomes simpler: we get for free a line segment length $\sqrt 2$ without having to construct it.
\end{frame}

\begin{frame}[plain]
 Metric geometry is usually treated analytically: described using a coordinate system and then manipulated using equations. The convenience of the coordinate system has lost some of the intuitively geometric feel of the straight-edge and compass geometry. 
 
 In 1932, George Birkhoff created a new way of describing metric geometry. He thought of axioms describing the ruler and protractor, which are simply marked variants of the straight-edge and compass. His axioms had a similarly intuitive base as Euclid's original postulates describing the straight-edge and compass.
 
 \end{frame}

\begin{frame}[plain] 
  A partial motivation for Birkhoff was to use his axioms as a new way of teaching geometry. Previously, students had to take many theorems for granted since it was too involved and time-consuming to prove them from the axioms. Birkhoff hoped that with his new axioms that had all the power of the real numbers, and were written at a high-level, students could immediately prove theorems from the axioms, and thus have a better idea of the method of mathematics. This property of the axioms also makes them perfect for a formalised theory of metric geometry, since the work required to prove interesting theorems is reduced. This talk is about a partial formalisation of Birkhoff's axioms for geometry in the proof assistant Isabelle.
  \end{frame}
  
\frame{\titlepage}

%Maybe we shouldn't begin with title page slide but instead a question (blank slide and ask question) or some such
\begin{frame}
\frametitle{Overview}
%\begin{itemize}

%\end{itemize} 
 
 \tableofcontents
 
\end{frame}

\section{Isabelle}
\begin{frame}
\frametitle{Isabelle}
 
 \begin{center}
 \graphicspath{{/Users/Imogen/Desktop/Birkhoff_Presentation/}}
\includegraphics[width=0.75\textwidth]{Logo-of-Isabelle_svg.pdf}
 \end{center}
 
\end{frame}

\begin{frame}
\frametitle{Isabelle}
 Isabelle is an interactive proof assistant. You can write axioms, definitions, theorems and proofs in Isabelle. Isabelle will check that the axioms, definitions and theorems make sense. It will also check that the proofs are correct. Every proof in Isabelle will be a formal proof. Isabelle can also automatically prove some simple statements.
 
\end{frame}

\section{Ruler and protractor axioms}
% Birkhoff has four main axioms in his paper and I'm going to describe three of them.

% We will look at Brossard's versions of the axioms because those are the versions I formalised.
{\graphicspath{{/Users/Imogen/Desktop/Birkhoff_Presentation/}}
    \usebackgroundtemplate{\includegraphics[keepaspectratio=true,width=1\paperwidth]{ruler_axiom.pdf} }

\begin{frame}
\frametitle{Ruler axiom}

 
\end{frame}
}

\begin{frame}
\frametitle{Ruler axiom}
 Birkhoff has four main axioms in his paper and I'm going to describe three of them. However, there are also some basic axioms describing lines, most of which Birkhoff omits mentioning. Although our formalisation is based on Birkhoff's ruler and protractor geometry, we actually follow Brossard's rewriting of Brikhoff's axioms, since it is more formal and we found Brossard's axioms easier to express in Isabelle. 
 
\end{frame}

\begin{frame}[fragile,plain]
\frametitle{Ruler axiom}
\begin{lstlisting}[language=Isar, mathescape = true]{}  
 locale Line_Measure = Lines isLine
  for isLine ::"'p set \<Rightarrow> bool" +
  fixes Coord :: "'p set \<Rightarrow> ('p \<Rightarrow> real) set" 
  assumes
  "l \<in> Line \<Longrightarrow> \<exists> x. x \<in> Coord l" 
  and
  "\<lbrakk>l \<in> Line; x \<in> Coord l\<rbrakk> 
       \<Longrightarrow> bij_betw x l (UNIV::real set)"
  and
  "l \<in> Line \<Longrightarrow> 
  \<lbrakk>x_i \<in> Coord l ;
   bij_betw x_j l (UNIV::real set)\<rbrakk>
   \<Longrightarrow> ((x_j \<in> Coord l) =  \<forall> A \<in> l. \<forall> B \<in> l. 
       \<bar>x_i A - x_i B\<bar> = \<bar>x_j A - x_j B\<bar> )"
\end{lstlisting}
 
\end{frame}

\begin{frame}
\frametitle{Ruler axiom}
\begin{itemize}
\item We formalise the ruler axiom in a \Icode!locale!. This is a structure in Isabelle which treats the statements written in it as assumptions for the following context. It can later be instantiated to a theory satisfying those statements. 
\item \Icode!Coord! is the set of all coordinate functions corresponding to a line.
\item The first locale assumption says that for each line, there is a coordinate function of that line.
\item The second assumption says that a coordinate function for a line forms a bijection between the line and the real numbers. So there will be a point on the line corresponding to $1$ and to $\pi$, and for every point, there will be a real number it represents.
\end{itemize}
\end{frame}

\begin{frame}
\frametitle{Ruler axiom}
\begin{itemize}
\item The final assumption says that a bijection between the line and the real numbers is a coordinate function if the distance between any two points $A$ and $B$, relative to it, is the same as the distance between the points relative to another coordinate function of the line. Intuitively it means that any two coordinate functions agree on lengths.
\end{itemize}
\end{frame}

{\graphicspath{{/Users/Imogen/Desktop/Birkhoff_Presentation/}}
    \usebackgroundtemplate{\includegraphics[keepaspectratio=true,width=1\paperwidth]{bundle_halfline_redblue.pdf} }
\begin{frame}
\frametitle{Lines are to points as bundles are to half-lines}

\end{frame}}
\begin{frame}
\frametitle{Lines are to points as bundles are to half-lines}
 A half-line is defined by two points: an endpoint, and a second point which tells us the direction in which the half-line extends. Any point on the half-line except the endpoint can be used as the second point. A bundle is a collection of half-lines with the same endpoint, much the same as a line is a collection of points.
\end{frame}

{\graphicspath{{/Users/Imogen/Desktop/Birkhoff_Presentation/}}
    \usebackgroundtemplate{\includegraphics[keepaspectratio=true,width=1\paperwidth]{protractor_axiom.pdf} }

\begin{frame}
\frametitle{Protractor axiom}
 
 
\end{frame}
}

\begin{frame}
\frametitle{Protractor axiom}
 The protractor axiom is the same as the ruler axiom except that the coordinated functions need only agree up to mod $2\pi$. The \Icode!locale! describing this axiom in Isabelle is analogous to the \Icode!locale! for the ruler axiom, so we will not describe it here.
 
\end{frame}

%modular arithmetic 
% It doesn't matter which unit we use for angles. Hence pi radians can be 180 degrees or 2 units as we use here. (It will mean there will be an extra constant in the formula relating circumference to radius of a circle).
% So we can just take the most convenient unit to formalise: arithmetic modulo 4. Since 4 is a small integer, it is easy to prove that every fourth number is divisible by four by using cases. 
% theorem
% four also intuitively describes the four quadrants that we divide angles into when we want to work out if the trigonometric functions are positive or negative.
\section{Continuity Axiom}

\begin{frame}
\frametitle{Continuity Axiom}
\begin{equation*}
 \angle AOP + \angle POB = \angle AOB \quad (\text{mod} \, 2 \pi)
 \end{equation*}
 \graphicspath{/Users/Imogen/Desktop/Birkhoff_Presentation/}
 \includegraphics[keepaspectratio=true,width=1\textwidth]{/Users/Imogen/Desktop/Birkhoff_Presentation/cont_axiom_pi_imp_straight}

\end{frame}

\begin{frame}
\frametitle{Continuity Axiom}
The Continuity Axiom has two directions. We will only need the first direction.

It states that given a bundle centred at $O$ and distinct noncollinear half-lines $OA$ and $OB$ in the bundle, then for every point $P$ between the points $A$ and $B$, there is a half-line with endpoint $O$ through $P$ such that the equation holds. 

 It gives a relation between the distance and angle measure since the equation $d(A,P) + d(P,B) = d(A,B)$, where $d(X,Y)$ is the distance between points $X$ and $Y$, becomes a similar equation in terms of the angle measure.

\end{frame}
\section{Measure of a straight line}
\begin{frame}
\frametitle{Measure of a straight line}
\begin{center} 
 We would now like to concentrate on the following theorem which was particularly interesting to formalise.
 
 \medskip
 
 An angle has measure $\pi$ iff it is a straight line.
 
 \medskip
 
 It is also the foundation for many other important theorems and definitions.
 \end{center}
 % motivation for why it is important
 
\end{frame}

{{\graphicspath{{/Users/Imogen/Desktop/Birkhoff_Presentation/}}
    \usebackgroundtemplate{\includegraphics[keepaspectratio=true,width=1\paperwidth]{plane_separation_hands.pdf} }
\begin{frame}
\frametitle{Plane separation}
 
 
 
 \btVFill
 {\footnotesize Hand picture by Cy21 [CC BY-SA 3.0  (https://creativecommons.org/licenses/by-sa/3.0)]}
\end{frame}}

\begin{frame}
\frametitle{Plane separation}
The theorem $\pi$ implies straight is important because it allows us to prove that a straight line separates the plane. Hence it also allows us to define the notions of right and left which are important for describing orientation and it allows us to define closed polygons.
\end{frame}

\begin{frame}
\frametitle{Measure is independent of bundle}
% the theorem also allows us to prove this
For noncollinear half-lines

\medskip

\begin{center} \Huge
 All protractors are the same!
% Equal implies zero
% Straight implies $\pi$
\end{center}

\bigskip

What about collinear half-lines?
\end{frame}

\begin{frame}
\frametitle{Measure is independent of bundle}
From the protractor axiom we find that for any particular bundle, the measure of an angle formed by two half-lines is the same. But we would like to know if the measure of the angle formed by two half-lines is the same no matter which bundle they are considered to be in. In effect, we are wondering if all protractors are measuring the angle the same. In fact, if the half-lines forming the angle are not collinear, then this follows immediately from the fact that there is only one bundle that they can be in. But what if they are not collinear? There are two possibilities. If the half-lines are equal, then the angle is zero no matter what bundle the half-line is contained in. That leaves the case in which the half-lines form a straight line. If we can prove the measure is always $\pi$, we have shown that all half-lines are the same.


\end{frame}

\begin{frame}[fragile]
\frametitle{$\pi$ implies straight}
  
  The statement of $\pi$ implies straight in Isabelle is as follows
  
 \begin{lstlisting}[language=Isar, mathescape = true]{}    
lemma pi_imp_straight: assumes "B_O \<in> Bundle" "OA \<in> B_O" 
"OP \<in> B_O" "\<phi> \<in> (Acoord B_O)" "ameasure_rel \<phi> OA OP = \<pi>" 
shows "\<exists> L \<in> Line. OA \<union> OP = L" 
\end{lstlisting}  
 
So we have two distinct half-lines $OA$ and $OP$ which form an angle with measure $\pi$. In our formal proof of the theorem we assume that they don't form a straight line with the aim of obtaining a contradiction. Since it is not a straight line we can construct a point $B$ on line $AP$ and not on line $OA$ so that $P$ is strictly between $A$ and $B$.
 
\end{frame}
\begin{frame}
\frametitle{Using continuity axiom twice}
Brossard says

\textit{... in the unique bundle $B_0$ containing the half-lines $OA$, $OB$, $OP$ the continuity axiom implies that $\angle AOP + \angle POB = \angle AOB \quad (\text{mod } 2\pi)$.} 

We end up using the Continuity Axiom once to give us that the three lines are in a unique bundle, and a second time to actually show that the identity holds. These two applications of the Continuity Axiom need to be done separately because the first application gives us angle measure relative to the new bundle, and the second gives us angle measure relative to the original bundle.

\end{frame}

{{{\graphicspath{{/Users/Imogen/Desktop/Birkhoff_Presentation/}}
    \usebackgroundtemplate{\includegraphics[keepaspectratio=true,width=1\paperwidth]{cont_axiom_pi_imp_straight}}
 \begin{frame}
 
 \end{frame}
 }
\begin{frame}
\frametitle{Obtaining a  contradiction}

All the angles are between $0$ and $\pi$ because we have defined them to be acute angles. We plug into the equation the value of angle $\angle AOP = \pi$, obtaining
\begin{equation*}
\pi + \angle POB = \angle AOB \quad (\text{mod } 2\pi)
 \end{equation*}
  % go to geogebra file
 We work out using modular arithmetic and previously proven facts that we can't have $\angle POB = 0$ and we can't have $\angle POB = \pi$. Hence, from the equation, we see $\angle AOB$ is not between $0$ and $\pi$ which is a contradiction. 
\end{frame}

{{{\graphicspath{{/Users/Imogen/Desktop/Birkhoff_Presentation/}}
    \usebackgroundtemplate{\includegraphics[keepaspectratio=true,width=1\paperwidth]{straight_imp_pi.pdf} }
\begin{frame}
\frametitle{Straight implies $\pi$}
 \vfill
 
 \bigskip
 
 \bigskip
 
 \bigskip
 
 \bigskip
  
 \bigskip
 
 \bigskip
 
 \bigskip
 
The idea of the proof of this direction is to show that the straight angle lies on the same line as an angle which is chosen to have measure $\pi$. It uses the direction $\pi$ implies straight as part of the proof. The formal proof follows Brossard's pen-and-paper proof quite closely.
 
\end{frame}
}
\section{Differences between Brossard's proof and formal proof}
\begin{frame}[fragile]
\frametitle{Differences between Brossard's proof and formal proof}
 \begin{itemize}
 \item We used locales to formalise the axioms which could allow instantiation to equivalent systems.
 
\item The proof of measure $\pi$ iff it is straight required one more application of the Continuity Axiom than Brossard explicitly states.

\item Overall, we found that there were almost a hundred lemmas which needed substantial proof in Isabelle, but were simply assumed by Birkhoff or Brossard. Here are a few examples. 
\end{itemize}
\end{frame}

\begin{frame}[fragile]
\frametitle{Differences between Brossard's proof and formal proof}

 \begin{lstlisting}[language=Isar, mathescape = true]{}    
lemma sameside_eq_notbetween: assumes 
"between A X B"  "between A X P" 
 shows "\<not> between B X P"
\end{lstlisting}

 
  \begin{lstlisting}[language=Isar, mathescape = true]{}    
lemma halfline_independence: assumes 
"B \<in> halfline X A" "B \<noteq> X" 
shows "halfline X A = halfline X B"
\end{lstlisting}

 \begin{lstlisting}[language=Isar, mathescape = true]{}      
lemma angle_at_origin: assumes "B \<in> Bundle" 
"\<phi> \<in> Acoord B"  "g\<in>B" "h\<in>B" 
shows "\<exists>f\<in>B. \<phi> g - \<phi> h =4 \<phi> f" 
\end{lstlisting}
\end{frame}

\section{Next steps}
\begin{frame}
\frametitle{Next steps}


\graphicspath{{/Users/Imogen/Desktop/Birkhoff_Presentation/}}
\tikz[remember picture, overlay] \node[anchor=center] at (current page.center) {\includegraphics[keepaspectratio=true,width=1\paperwidth]{trig.pdf}};

 % there is one more axiom in Birkhoff's system: it is concerning the similarity of triangles and will make the system Euclidean. It is the side-angle-side theorem familiar from school
 
% proofs of the theorems on triangles. 

%Then moving beyond Brossard's paper, but possibly following his book written with Beatley, theorems of circles could be formalised and finally this could be applied to a definition of sine and cosine from geometry. 
 
\end{frame}

\begin{frame}
\frametitle{Next steps}
There is one more axiom in Birkhoff's system concerning the similarity of triangles and this axiom will make the system Euclidean. It is the Side-Angle-Side theorem which may be familiar from school. Theorems on triangles can then be proven.

Then moving beyond Birkhoff's original paper on the ruler and protractor postulates, but possibly following his book written with Beatley, theorems of circles could be formalised and finally this could be applied to a definition of sine and cosine from geometry. 
\end{frame}

\end{document}